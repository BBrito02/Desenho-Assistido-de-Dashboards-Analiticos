%!TEX root = ../template.tex
%%%%%%%%%%%%%%%%%%%%%%%%%%%%%%%%%%%%%%%%%%%%%%%%%%%%%%%%%%%%%%%%%%%%
%% chapter3.tex
%% NOVA thesis document file
%%
%% Capítulo 3 (revisto após feedback dos orientadores)
%%%%%%%%%%%%%%%%%%%%%%%%%%%%%%%%%%%%%%%%%%%%%%%%%%%%%%%%%%%%%%%%%%%%

\typeout{NT FILE chapter3.tex}%

\chapter{Abordagem Dashboard Designer}
\label{cha:dashboard_designer_abordagem}

As ferramentas existentes para a análise e desenho de \textit{dashboards} apresentam limitações no que toca ao suporte no processo de desenvolvimento iterativo de design, à exploração de alternativas e à recolha de \textit{feedback} de forma contextualizada. No âmbito desta dissertação, pretende-se desenvolver o \textit{Dashboard Designer}, uma ferramenta web que suporta a prototipagem e documentação de \textit{dashboards} interativos com base na linguagem \gls{IVML}.

Este capítulo descreve a forma como o problema é abordado e como o desenvolvimento da ferramenta é planeado e estruturado. Em primeiro lugar, é apresentada a abordagem global adotada, destacando opções arquiteturais e os princípios funcionais da ferramenta ao nível conceptual. De seguida, é descrito o processo de desenvolvimento, incluindo a organização do trabalho em etapas, mecanismos de validação com os orientadores e uma síntese das funcionalidades implementadas. Posteriormente, é apresentado o meta-modelo que fundamenta a ferramenta. Por fim, são descritas as tecnologias e ferramentas selecionadas para suportar a implementação.

\section{Abordagem}
\label{sec:cap3_abordagem}

A abordagem adotada para o desenvolvimento do \textit{Dashboard Designer} é orientada para suportar um processo iterativo de prototipagem, onde diferentes intervenientes (por exemplo, \textit{designers}, e utilizadores finais) conseguem construir, analisar e refinar \textit{dashboards} de forma progressiva. Para isso, a solução é desenhada para representar componentes e relações de forma visual e explícita, reduzindo ambiguidades na comunicação e preservando o contexto associado a cada decisão de design.

Do ponto de vista arquitetural, a ferramenta é concebida como uma \gls{SPA}, executada no navegador, privilegiando uma experiência de utilização fluída e responsiva. Esta opção permite reduzir fricção de acesso (sem instalação dedicada) e simplificar a distribuição da ferramenta. Adicionalmente, a solução é planeada para funcionar sem dependência de um \textit{backend} dedicado, centrando-se numa abordagem \textit{client-side} para armazenamento e portabilidade dos projetos, o que facilita a utilização em diferentes contextos e ambientes.

A aplicação é estruturada em dois modos principais de utilização:
\begin{itemize}
  \item \textbf{Editor Mode}: destinado à criação e edição de protótipos, permitindo ao utilizador construir a estrutura do \textit{dashboard} através de nós e ligações, e configurar propriedades relevantes para cada componente.
  \item \textbf{Review Mode}: orientado para análise e recolha de \textit{feedback}, com foco na inspeção do protótipo e na comunicação contextualizada sobre elementos concretos (nós e ligações).
\end{itemize}

Ao nível conceptual, a ferramenta é planeada para suportar funcionalidades essenciais para a prototipagem e documentação de \textit{dashboards}, incluindo: uma abordagem visual baseada em grafo (nós/arestas), mecanismos de \textit{drag-and-drop} para criação rápida de protótipos, integração de \textit{screenshots} e elementos visuais, validação de dependências entre componentes, e suporte à recolha de \textit{feedback} diretamente sobre os elementos representados.

\section{Processo de Desenvolvimento}
\label{sec:cap3_plano_desenvolvimento}

O desenvolvimento do \textit{Dashboard Designer} é conduzido de forma incremental, através de iterações regulares que permitem validar decisões e ajustar a implementação de forma progressiva. Esta abordagem favorece a identificação precoce de problemas de usabilidade e consistência conceptual, e facilita a integração de melhorias ao longo do tempo.

Ao longo do desenvolvimento são realizados \gls{PDS} semanais com os orientadores. Nestes encontros é apresentado o progresso realizado na semana, as principais dificuldades encontradas e o plano de trabalho proposto para a semana seguinte. Este modelo de acompanhamento promove uma validação contínua das decisões tomadas e assegura alinhamento com os objetivos definidos para a dissertação.

A Tabela~\ref{tab:cap3_plano_fases} resume as etapas consideradas no planeamento do desenvolvimento, clarificando objetivos e entregáveis por fase.

\begin{table}[htbp]
\centering
\caption{Síntese do plano de desenvolvimento do \textit{Dashboard Designer}.}
\label{tab:cap3_plano_fases}
\begin{tabular}{p{0.16\textwidth} p{0.38\textwidth} p{0.38\textwidth}}
\hline
\textbf{Etapa} & \textbf{Objetivo} & \textbf{Resultados previstos} \\
\hline
Levantamento de requisitos
& Consolidar necessidades funcionais e requisitos de usabilidade para a ferramenta.
& Lista estruturada de requisitos (funcionais e não-funcionais) que guiam o desenvolvimento, bem como as funcionalidades implementadas. \\
\hline
Meta-modelo e protótipos
& Definir a estrutura conceptual do domínio e desenhar protótipos que orientam a interface.
& Meta-modelo (Sec.~\ref{sec:meta_modelo}) e protótipos iniciais para validação visual e estrutural. \\
\hline
Implementação visual
& Construir a base visual do editor (canvas, nós, menus) alinhada com os protótipos.
& Interface funcional do Editor Mode ao nível visual (canvas e representação dos nós). \\
\hline
Implementação lógica
& Transferir regras do meta-modelo para a ferramenta (atributos, ligações, dependências e validações).
& Estrutura semântica operacional (dados por nó, regras de ligação e coerência do grafo). \\
\hline
Funcionalidades diferenciadoras
& Implementar capacidades que suportam iteração e colaboração.
& Review Mode para recolha de feedback e suporte a alternativas/perspetivas nos nós. \\
\hline
Validação e testes
& Verificar consistência, robustez e adequação do comportamento esperado.
& Testes manuais e validações contínuas através da criação de protótipos representativos. \\
\hline
\end{tabular}
\end{table}

A fase inicial de levantamento de requisitos permitiu consolidar as funcionalidades que a ferramenta deve suportar, assegurando que o desenvolvimento permanece alinhado com o objetivo central da dissertação: apoiar a conceção, prototipagem e revisão de \textit{dashboards} interativos. De seguida, apresentam-se os requisitos funcionais do \textit{Dashboard Designer}, cujo detalhe de funcionamento é aprofundado no Capítulo~\ref{cha:arquitetura_implementacao}.

\begin{itemize}
  \item \textbf{Abordagem \textit{drag-and-drop}} para criação e reorganização rápida de componentes no \textit{canvas}.
  \item \textbf{Integração de \textit{screenshots}} como suporte visual à representação do \textit{dashboard}, facilitando comunicação e validação de layout.
  \item \textbf{Alternativas de layout do \textit{dashboard}}, permitindo explorar e comparar diferentes variações do protótipo.
  \item \textbf{Atalhos de produtividade (\textit{undo/redo})}, reduzindo fricção durante a edição e experimentação de alternativas.
  \item \textbf{Suporte a diferentes perfis de utilizador}, adaptando o fluxo de utilização a intervenientes com objetivos distintos.
  \item \textbf{Representação alinhada com o meta-modelo}, assegurando consistência conceptual entre elementos e relações.
  \item \textbf{Visão detalhada por elemento}, disponibilizando informação estruturada sobre cada componente (por exemplo, tipo de visualização, dados e objetivos).
  \item \textbf{Recolha de \textit{feedback} integrada}, permitindo comentários contextualizados durante a análise de protótipos.
\end{itemize}

\section{Meta-modelo} % (fold)
\label{sec:meta_modelo}

Com o principal objetivo de apoiar os \textit{designers} no processo de conceção e prototipagem de \textit{dashboards} interativos, foi desenvolvido um meta-modelo que representa de forma estruturada a linguagem IVML, assim como as suas entidades e as relações existentes entre essas entidades. Este meta-modelo irá ser utilizado como uma base conceptual para a ferramenta proposta, permitindo a representação tanto das componentes visuais e interativas de um \textit{dashboard}, como as suas componentes analíticas que sustentam a sua construção.

O desenho e estrutura do meta-modelo foi pensado para englobar não só os elementos visuais utilizados nas representações, como visualizações, interações, e \textit{tooltips}, mas também para incluir as fases iniciais que suportam o processo de criação de \textit{dashboards}, como a definição do âmbito de análise, a formulação de questões analíticas, e a identificação das perspetivas e objetivos de análise. Esta abordagem mais abrangente de elementos tem como objetivo mostrar que o processo de \textit{design} de \textit{dashboards} interativos não se foca apenas na representação gráfica dos seus elementos, mas também na definição da componente analítica que justifica a sua criação.

É apresentado em apêndice o meta-modelo completo, que engloba todas as diferentes componentes numa só representação (ver apêndice~\ref{app:meta_modelo_app}). Nas secções seguintes, vai ser feita a descrição detalhada das diferentes componentes que definem o meta-modelo. Esta abordagem de separação do meta-modelo em diferentes componentes de menor amplitude facilita a sua apresentação e compreensão.

\subsection{Componente Analítica e Estrutural} % (fold)
\label{sub:anal_struct_comp}

Nesta secção serão exploradas em detalhe as duas componentes centrais que definem o meta-modelo: a componente analítica e a componente estrutural. A organização conceptual destas componentes pode ser observada na Figura~\ref{fig:comp_anal_struct}. Adicionalmente, a componente analítica inclui exemplos integrados na sua representação, baseados em casos concretos retirados da dissertação de Ana Milroy~\cite{milroy2025}, de forma a facilitar a interpretação da mesmo.

A componente analítica tem como principal objetivo representar o processo inicial de definição conceptual e estratégica que orienta a criação dos diferentes \textit{dashboards}. Esta componente é composta por diferentes entidades que, de forma conjunta, tornam o processo de conceptualização possível. Cada entidade presente na componente descrita desempenha um papel fundamental para a correta implementação dos conceitos analíticos necessários para a conceção de uma \textit{dashboard}. O processo tem início na entidade \textbf{Analysis Scope}, que define o âmbito geral da análise e estabelece o contexto onde se vão inserir os \textit{dashboards} a desenvolver. Esta entidade constitui o ponto de partida de todo o raciocínio analítico, uma vez que é através da estipulação do âmbito que se definem os restantes elementos. Associada ao âmbito, surge a entidade de \textbf{Datum}, que, em conjunto com a entidade \textbf{Data Collection} definem o conjunto de dados disponíveis a serem utilizados para responder às diferentes necessidades analíticas. A entidade \textbf{Question Collection} agrupa o conjunto de questões analíticas que se pretendem ver respondidas com a implementação dos diferentes \textit{dashboards}. Estas questões são formuladas com base no âmbito de análise em que estão inseridas e devem estar alinhadas com as necessidades dos utilizadores. Dentro de cada âmbito podem ser definidos diferentes \textbf{Analysis Themes}, que representam as áreas específicas no âmbito que pretendemos analisar. Cada tema de análise pode ser analisado sob diferentes \textbf{Analysis Perspective} que se referem a diferentes formas de analisar e representar os temas a que estão associados. Por fim, cada perspetiva está associada a um ou mais \textbf{Analysis Objective}, que definem as metas que se pretendem vir a ser atingidas através das diferentes visualizações que irão compor os \textit{dashboards}.

A segunda componente representada no diagrama é a componente estrutural, cujo principal objetivo é representar os diferentes elementos visuais e interativos que vão ser utilizados na representação dos diferentes \textit{dashboards} a desenvolver. A entidade que desempenha um papel central nesta componente é a \textbf{Dashboard}, responsável por agregar todos os elementos que compõem a interface visual. Cada \textit{dashboard} pode integrar múltiplas entidades \textbf{Visualization}, que correspondem às representações gráficas dos dados a serem analisados (como gráficos, mapas ou tabelas). Uma visualização pode estar associada a zero o mais entidades \textbf{Caption}, que representam a legenda da informação que está a ser mostrada nas diferentes visualizações. Cada visualização pode estar associada a zero ou mais entidades \textbf{Tooltip}, estas entidades fornecem informação adicional ao utilizador sobre um ou mais elementos presentes na visualização. A entidade \textbf{Placeholder} pode estar associada tanto à visualização como ao próprio \textit{dashboard}, e serve para representar componentes estáticas como imagens ou títulos. Esta entidade contribui para estruturar o \textit{display} da interface, sem sobrecarregar o utilizador com informação analítica. Por fim, é introduzida a entidade \textbf{InteractionComp}, uma classe abstrata que agrega diferentes tipos de componentes interativos, com comportamentos distintos. A primeira entidade é \textbf{Button}. Esta entidade é utilizada para ilustrar botões acionáveis responsáveis por executar ações específicas. A segunda entidade é \textbf{Parameter}, usada quando se pretende introduzir parâmetros que influenciam diretamente o conteúdo a ser mostrado. Finalmente, é introduzida a entidade \textbf{Filter}, destinada à representação de filtros aplicáveis a uma ou mais visualizações, permitindo refinar os dados exibidos conforme os critérios definidos pelo utilizador.

Como é possível observar na Figura~\ref{fig:comp_anal_struct}, são estabelecidas ligações diretas de dependência e agregação entre entidades da componente estrutural e entidades da componente analítica, sendo esta a razão pela qual estas componentes são apresentadas de forma simultânea e não de forma separada. Estas ligações entre componentes descrevem o comportamento e relação funcional que estas componentes têm em comum. Um exemplo destas ligações é a relação entre a entidade \textbf{Analysis Perspective} e a entidade \textbf{Dashboard}. Como é descrito anteriormente, cada tema de análise pode englobar uma ou mais perspetivas de análise, sendo que estas perspetivas de análise são representadas visualmente através da criação de diferentes tipos de \textit{dashboard}. Da mesma forma, a entidade \textbf{Analysis Objective} está diretamente ligada à entidade \textbf{Visualization}. Cada perspetiva de análise pode agregar um ou mais objetivos de análise, sendo que estes objetivos são respondidos através da criação de diferentes tipos de visualização. Estas ligações entre entidades são a ponte que estabelece a ligação entre a componente analítica e estrutural, dando contexto às diferentes componentes visuais implementadas nas representações.

\begin{figure}[htbp]
  \includegraphics[width=\textwidth]{meta_modelo/compAnlyticalStructural_cropped(1)}
  \centering
  \caption{Componente Analítica e Estrutural do meta-modelo.}
  \label{fig:comp_anal_struct}
\end{figure}

\subsection{Atributos dos Dados e Atributos Visuais} % (fold)
\label{sub:data_vis_attr}

Depois de definirmos a componente analítica e a componente estrutural, isto é, depois de definirmos o que será analisado e como iremos estruturar o \textit{dashboard}, é essencial definirmos como é que os dados vão ser representados visualmente. É neste contexto que entra a componente de dados e atributos visuais, esta componente tem como principal objetivo estabelecer a ligação entre os dados e os elementos gráficos que os representam. Esta componente encontra-se representada na Figura~\ref{fig:comp_data_vis}.

A componente de dados e atributos visuais tem como principal objetivo a representação gráfica dos dados de forma a torná-los de fácil compreensão. Desta forma, as duas entidades que apresentam um papel central nesta componente são: \textbf{Visualization}, a entidade responsável por fazer a conversão direta dos dados em representações visuais de fácil compreensão, e \textbf{Datum}, a entidade que representa os dados a serem utilizados nas diferentes visualizações. Cada visualização pode representar os dados através de diferentes tipos de representações gráficas. A classe abstrata \textbf{Graph Type} agrupa todos os tipos de representação gráfica que uma visualização pode tomar (como gráficos de barras, linhas, mapas, entre outros). Para ajudar a representação, as visualizações podem ter associadas \textbf{Visual Variables} e \textbf{Captions}. As variáveis visuais são utilizadas para distinguir métricas diferentes dentro de uma mesma visualização, através da utilização de elementos como cor, forma ou tamanho. As legendas, por sua vez, servem para fornecer contexto à informação apresentada nas diferentes visualizações. Já mencionadas na componente estrutural, as legendas são cruciais para garantir que o utilizador compreenda completamente o que está a ser mostrado.

\begin{figure}[htbp]
  \includegraphics[width=\textwidth]{meta_modelo/dataVis}
  \centering
  \caption{Componente de Dados e Atributos Visuais do meta-modelo.}
  \label{fig:comp_data_vis}
\end{figure}

\subsection{Componente de Interação} % (fold)
\label{sub:int_diagram}

A componente de interação permite representar os diferentes elementos responsáveis por promover a interatividade entre o utilizador e o \textit{dashboard}. É através desta componente que é possível transformar um \textit{dashboard} estático e pouco interativo numa solução dinâmica, interativa e de fácil utilização, onde é dada ao utilizador a possibilidade de navegar por diferentes perspetivas e compreender melhor as relações que as diferentes componentes estabelecem entre si. A representação visual desta componente encontra-se ilustrada na Figura~\ref{fig:interaction}.

Esta componente é modelada a partir da classe abstrata \textbf{Interaction Comp}, que representa os elementos que introduzem comportamento no \textit{dashboard}, nomeadamente \textbf{Button}, \textbf{Parameter}, e \textbf{Filter}. Estas entidades estão sempre associadas a um \textbf{Interaction Type}, que define a forma como a interação vai ser desencadeada, por exemplo, através de um \textbf{Click} ou de um \textbf{Hover}. De entre as três entidades de interação, os parâmetros e filtros desempenham um papel direto na manipulação dos dados. Estas duas entidades são ligadas à entidade \textbf{Data Action} que representa a ação que será aplicada sobre os dados. Cada ação é classificada segundo um \textbf{Action Type}, determinando o efeito que terá na visualização dos dados, podendo este ter efeito de \textbf{Highlight} ou de \textbf{Filtering}. Estes tipos de ações resultam num impacto direto sobre as várias entidades visuais, como \textbf{Visualization}, \textbf{Caption}, e \textbf{Dashboard}, modificando a informação apresentada ao utilizador com base nas diferentes interações realizadas. A entidade \textbf{Datum}, para além de representar os diferentes dados presentes nas diferentes entidades, pode ser usada para ativar \textbf{Tooltips} através de um tipo de interação associado. Desta forma, a componente de interação desempenha um papel essencial no reforço da experiência analítica do utilizador. 

\begin{figure}[htbp]
  \includegraphics[width=\textwidth]{meta_modelo/interaction}
  \centering
  \caption{Componente de Interação do meta-modelo.}
  \label{fig:interaction}
\end{figure}

\subsection{Componente de Navegação} % (fold)
\label{sub:nav_diagram}

A última componente criada no âmbito do meta-modelo é a componente de navegação. Esta componente representa as diferentes formas que o utilizador possui de forma a mudar de contexto analítico através de navegação entre outros \textit{dashboards} ou através de ligações externas. A representação visual desta componente encontra-se ilustrada na Figura~\ref{fig:navigation}.

A entidade central neste tipo de componente é a classe abstrata \textbf{Navigation Comp}, que representa os diferentes tipos de navegação possíveis. Através desta classe, é possível definir dois comportamentos distintos:

\begin{itemize}
    \item \textbf{Dashboard Nav} que representa a navegação interna entre \textit{dashboards};
    \item \textbf{Nav Link} que representa a navegação externa, como por exemplo, para páginas web;
\end{itemize}

De forma semelhante à componente de interação, a componente de navegação está sempre associada a um \textbf{Interaction Type} que define a forma como a navegação vai ser desencadeada - nomeadamente por um \textbf{Click} ou por um \textbf{Hover}. Esta componente de navegação pode estar associada a diferentes tipos de elementos visuais, tais como, \textbf{Button} e \textbf{Datum}. Esta integração permite que o utilizador integre facilmente mecanismos de navegação nas diferentes representações visuais, tornando a navegação mais simples e adaptada às necessidades do utilizador.

\begin{figure}[htbp]
  \includegraphics[width=\textwidth]{meta_modelo/navigation}
  \centering
  \caption{Componente de Navegação do meta-modelo.}
  \label{fig:navigation}
\end{figure}

\section{Tecnologias e Ferramentas}
\label{sec:cap3_tecnologias}

Para suportar a implementação do \textit{Dashboard Designer}, são selecionadas tecnologias e bibliotecas com foco em: desenvolvimento rápido, modularidade, suporte a interfaces interativas complexas e capacidade de persistência e portabilidade de projetos. A seleção tecnológica procura equilibrar três dimensões essenciais: (i) \textbf{produtividade} durante o desenvolvimento, (ii) \textbf{robustez} e facilidade de manutenção do código, e (iii) \textbf{adequação ao domínio} de um editor visual baseado em nós e ligações.

Nas subsecções seguintes, apresentam-se as tecnologias selecionadas, agrupadas entre tecnologias base de desenvolvimento e bibliotecas mais específicas para suportar funcionalidades críticas do domínio, como \textit{drag-and-drop}, persistência local e exportação/importação de projetos.

\subsection{Tecnologias Base}
\label{subsec:cap3_tecnologias_base}

As tecnologias base foram escolhidas para garantir uma fundação sólida para o desenvolvimento de uma aplicação web moderna, com uma interface altamente interativa e facilmente extensível. Em particular, a combinação de \textit{React} e \textit{TypeScript} permite construir uma interface modular e reativa, enquanto assegura maior segurança e consistência na gestão de estruturas de dados complexas (por exemplo, diferentes tipos de nós, propriedades e configurações). Em paralelo, ferramentas como o \textit{Vite} e o \textit{Tailwind CSS} facilitam um ciclo de desenvolvimento rápido e uma construção eficiente de interfaces, aspetos importantes num processo iterativo e orientado à prototipagem.

De forma geral, estas tecnologias suportam uma implementação incremental, permitindo iterar rapidamente sobre a interface e o comportamento da aplicação, ao mesmo tempo que promovem uma base de código organizada, legível e sustentável.

\begin{itemize}
  \item \textbf{React}: \textit{framework} JavaScript para construção da interface gráfica de forma modular e reativa. Permite organizar a aplicação em componentes reutilizáveis e gerir atualizações do estado de forma eficiente, suportando a natureza dinâmica de um editor interativo~\cite{react}.
  \item \textbf{TypeScript}: \textit{superset} de JavaScript que introduz tipagem estática e verificação de tipos em tempo de compilação. Esta característica facilita a deteção de erros, melhora a legibilidade do código e aumenta a robustez na gestão de estruturas de dados complexas, relevantes num sistema com múltiplos tipos de nós e propriedades~\cite{typescript}.
  \item \textbf{Vite}: ferramenta de \textit{build} e desenvolvimento que acelera o ciclo de implementação através de arranque rápido e \textit{hot reload}. Esta abordagem melhora a produtividade ao permitir iterar rapidamente sobre alterações na interface e lógica do editor~\cite{vite}.
  \item \textbf{Tailwind CSS}: \textit{framework} CSS que permite criar interfaces responsivas e consistentes com menor sobrecarga de CSS manual. A utilização de classes utilitárias facilita a manutenção do estilo e acelera a prototipagem de componentes visuais~\cite{tailwindcss}.
\end{itemize}

\subsection{Bibliotecas Principais}
\label{subsec:cap3_bibliotecas_principais}

Para além das tecnologias base, a implementação do \textit{Dashboard Designer} exige bibliotecas que suportem diretamente funcionalidades centrais do domínio: criação de elementos por interação direta, manipulação de estruturas visuais com múltiplos componentes, e mecanismos de persistência e portabilidade de projetos sem dependência de infraestrutura externa. Por esta razão, são selecionadas bibliotecas orientadas a necessidades específicas, como \textit{drag-and-drop}, empacotamento de projetos para exportação e armazenamento local eficiente.

Estas bibliotecas permitem que a ferramenta ofereça uma experiência centrada no utilizador, reduzindo fricção durante a criação de protótipos e assegurando que projetos com recursos pesados (por exemplo, imagens) podem ser geridos de forma fluída no navegador, mantendo a aplicação responsiva.

\begin{itemize}
  \item \textbf{dnd-kit}: biblioteca para suportar padrões de \textit{drag-and-drop}, permitindo construir interações de arrastar e largar de forma controlada e extensível, especialmente útil na criação de nós a partir de um menu de componentes~\cite{dndkit}.
  \item \textbf{JSZip}: biblioteca para criação e manipulação de ficheiros \texttt{.zip} no navegador. Permite empacotar a configuração de um projeto e recursos associados (por exemplo, imagens), suportando exportação/importação de forma simples e portável~\cite{jszip}.
  \item \textbf{IndexedDB}: API nativa do navegador para armazenamento local estruturado. É relevante para persistir projetos e suportar recursos pesados (por exemplo, imagens) sem depender de um servidor, contribuindo para melhor desempenho durante operações de edição~\cite{IndexedDB}.
\end{itemize}

\subsubsection{React Flow}
\label{subsubsec:cap3_reactflow}

O \textit{React Flow} é uma biblioteca para construção de interfaces baseadas em grafos, suportando a criação e manipulação de nós e arestas em ambientes interativos~\cite{reactflow}. A sua escolha está diretamente associada ao paradigma do \textit{Dashboard Designer}: representar componentes e relações de um \textit{dashboard} de forma visual, explícita e manipulável. Ao fornecer uma abstração sólida para o \textit{canvas} e para operações comuns (como movimentação, ligação e renderização de elementos), o \textit{React Flow} reduz complexidade de implementação e permite concentrar o desenvolvimento nas regras e necessidades específicas do domínio.

Adicionalmente, a possibilidade de personalizar nós e arestas é essencial para adaptar a interface ao meta-modelo proposto, permitindo que diferentes tipos de elementos sejam representados com semânticas próprias, sem perder consistência de interação no editor.

Os principais aspetos que justificam a escolha do \textit{React Flow} incluem:
\begin{itemize}
  \item \textbf{Representação e manipulação de grafos}: suporte direto a nós e ligações, adequado à modelação de componentes e dependências.
  \item \textbf{Interação no canvas}: funcionalidades de navegação (como \textit{zoom} e \textit{pan}) para explorar protótipos com diferentes níveis de complexidade.
  \item \textbf{Personalização}: possibilidade de definir nós e arestas personalizados, permitindo adequar a representação aos elementos do meta-modelo.
  \item \textbf{Escalabilidade e desempenho}: mecanismos de renderização e interação preparados para cenários com múltiplos elementos no canvas.
\end{itemize}
