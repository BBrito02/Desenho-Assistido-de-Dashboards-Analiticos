%!TEX root = ../template.tex
%%%%%%%%%%%%%%%%%%%%%%%%%%%%%%%%%%%%%%%%%%%%%%%%%%%%%%%%%%%%%%%%%%%%
%% chapter4.tex
%% NOVA thesis document file
%%
%% Chapter with lots of dummy text
%%%%%%%%%%%%%%%%%%%%%%%%%%%%%%%%%%%%%%%%%%%%%%%%%%%%%%%%%%%%%%%%%%%%

\typeout{NT FILE chapter4.tex}%

\chapter{Abordagem e Processo de Desenvolvimento}
\label{chap:abordagem}

Este capítulo descreve a abordagem seguida na segunda fase da dissertação, centrada na materialização do \textit{Dashboard Designer} como uma ferramenta funcional e utilizável. Pretende-se contextualizar a forma como o problema foi endereçado do ponto de vista prático, explicando a estratégia adotada para planear e conduzir o desenvolvimento, bem como as decisões que orientaram a implementação ao longo do tempo.

Numa primeira parte, apresenta-se a abordagem ao problema e o planeamento da implementação, explicitando como foram definidas prioridades, marcos e critérios de sucesso. De seguida, descreve-se o processo de desenvolvimento, com ênfase numa metodologia iterativa e incremental, onde o sistema foi evoluindo por ciclos sucessivos de implementação e validação. Por fim, identificam-se funcionalidades não implementadas e limitações, discutindo-se as razões para a sua exclusão e estabelecendo-se a ponte para o capítulo seguinte, onde é detalhada a implementação e arquitetura da ferramenta.
